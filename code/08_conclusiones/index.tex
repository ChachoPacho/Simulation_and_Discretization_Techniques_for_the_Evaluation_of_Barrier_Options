\chapter{Conclusiones y Trabajo Futuro}
\label{chapter:conclusiones}

% OBJETIVO: Cerrar la tesis
% MUY IMPORTANTE - tribunal lee esto primero junto con intro

\section{Resumen de Resultados}
\label{sec:resumen_resultados}

% QUÉ ESCRIBIR AQUÍ (2-3 párrafos):
% - Qué se logró en esta tesis
% - Qué métodos se implementaron
% - Principales hallazgos

\section{Conclusiones Principales}
\label{sec:conclusiones_principales}

% QUÉ ESCRIBIR AQUÍ:
% Lista concreta de conclusiones

\subsection{Sobre los Métodos para Opciones Europeas}

% Concluir:
% - Fórmulas de R&R: ...
% - Árboles con Mallas: ...
% - Portafolio Replicante: ...
% - Cuándo usar cada uno

\subsection{Sobre Monte Carlo para Opciones Americanas}

% Concluir:
% - Desempeño del método
% - Precisión lograda
% - Tiempo computacional

\subsection{Comparación Global}

% Concluir:
% - Qué método es mejor en general
% - Trade-offs observados
% - Recomendaciones finales

\section{Limitaciones del Trabajo}
\label{sec:limitaciones}

% QUÉ ESCRIBIR AQUÍ:
% Ser honesto sobre qué no se hizo

% Ejemplos:
% - Solo volatilidad constante
% - No se consideraron saltos
% - Solo barreras simples (no dobles)
% - Número limitado de casos de prueba
% - No se optimizó código para máxima velocidad

\section{Trabajo Futuro}
\label{sec:trabajo_futuro}

% QUÉ ESCRIBIR AQUÍ:
% Extensiones y direcciones futuras

\subsection{Extensiones del Modelo}

% Proponer:
% - Barreras dobles
% - Barreras dependientes del tiempo
% - Volatilidad estocástica (Heston)
% - Modelos con saltos (Merton, Kou)
% - Dividendos discretos

\subsection{Extensiones de los Métodos}

% Proponer:
% - Otros métodos numéricos (cuasi-Monte Carlo, etc.)
% - Optimización de código (paralelización, GPU)
% - Cálculo eficiente de griegas de alto orden
% - Métodos híbridos

\subsection{Aplicaciones Prácticas}

% Proponer:
% - Implementar en sistema de trading real
% - Calibración con datos de mercado
% - Gestión de riesgo de portafolios
% - Pricing de productos estructurados

\subsection{Validación con Datos Reales}

% Proponer:
% - Comparar con precios de mercado
% - Estudiar smile de volatilidad
% - Backtesting de estrategias
