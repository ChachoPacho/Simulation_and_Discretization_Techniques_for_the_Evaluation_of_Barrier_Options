\chapter{Antecedentes}
\label{chapter:antecedentes}

\section{Opciones Barrera}
Las opciones tipo barrera son derivados financieros cuya existencia depende de si el precio del activo subyacente cruza o no un nivel de precio predeterminado (la barrera) durante el periodo de vida de la opción.

\subsection{Clasificación y Payoffs}
\begin{itemize}
    \item \textbf{Knock-in}: La opción vale sólo si se alcanzó la barrera.
    \begin{itemize}
        \item Up-and-In (UI): La barrera $B$ está por encima del precio inicial $S_0$.
        \item Down-and-In (DI): La barrera $B$ está por debajo del precio inicial $S_0$.
    \end{itemize}
    \item \textbf{Knock-out}: La opción deja de valer si se alcanzó la barrera.
    \begin{itemize}
        \item Up-and-Out (UO): La barrera $B$ está por encima del precio inicial $S_0$.
        \item Down-and-Out (DO): La barrera $B$ está por debajo del precio inicial $S_0$.
    \end{itemize}
\end{itemize}

Formalmente, el payoff de una opción barrera europea con vencimiento $T$, precio de ejercicio $K$ y barrera $B$ puede expresarse utilizando la función indicadora $\mathbb{I}_{A}$. Por ejemplo, para una opción \textit{Down-and-Out Call}:

\begin{equation}
    \text{Payoff}_{DO} = \max(S_T - K, 0) \cdot \mathbb{I}_{\{\min_{0 \leq t \leq T} S_t > B\}}
\end{equation}

Y para una opción \textit{Down-and-In Call}:

\begin{equation}
    \text{Payoff}_{DI} = \max(S_T - K, 0) \cdot \mathbb{I}_{\{\min_{0 \leq t \leq T} S_t \leq B\}}
\end{equation}

De manera análoga se definen los payoffs para las variantes \textit{Up} considerando el máximo del proceso en lugar del mínimo: $\mathbb{I}_{\{\max_{0 \leq t \leq T} S_t < B\}}$ para \textit{Up-and-Out} y $\mathbb{I}_{\{\max_{0 \leq t \leq T} S_t \geq B\}}$ para \textit{Up-and-In}.

\section{Dinámica de Precios y Modelado}
\subsection{Movimiento Browniano Geométrico}
El modelo estándar de valoración asume que el precio del activo subyacente $S_t$ sigue un proceso...

\subsection{Primer Tiempo de Pasaje}
Para las opciones barrera, es fundamental el estudio del primer instante en que el proceso alcanza el nivel $B$...

\completar{Desarrollar definiciones matemáticas y fórmulas de payoff}
