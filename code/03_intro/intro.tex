\chapter{Introducción}
\label{chapter:intro}

\completar{INTRO}

\section{Resultados originales presentados}

\completar{SI HAY RESULTADOS COMO CONFERENCIAS O PAPERS}

\section{Organización del trabajo}

\section{Organización del trabajo}

Los capítulos de este trabajo se organizan de la siguiente manera:

En el Capítulo~\ref{chapter:antecedentes} se 
presentan los fundamentos de sistemas de riego en general, su uso, y por que importante detectarlos, y como se esta trabajando actualmente con redes neuronales para lograrlo.

El Capítulo~\ref{chapter:} describe el estado de [SISTEMA\_OBJETO], detallando las funcionalidades, la arquitectura y las limitaciones presentes en [IMPLEMENTACION\_ACTUAL].

Los Capítulos~\ref{chapter:} y~\ref{chapter:} documentan el desarrollo de [NUEVA\_VERSION]. Abordan [ASPECTO\_TECNICO\_1] y [ASPECTO\_TECNICO\_2], respectivamente.

En el Capítulo~\ref{chapter:} se presentan los resultados obtenidos en términos de [METRICA\_1] y [METRICA\_2].

Finalmente, las conclusiones generales son presentadas en el Capítulo~\ref{chapter:conclu}, donde también se discuten posibles líneas futuras de desarrollo.

% =============================================================================
% EJEMPLO COMPLETADO:
% =============================================================================
% Los capítulos de este trabajo se organizan de la siguiente manera:
% 
% En el \autoref{chapter:fundamentos} se presentan los fundamentos de inteligencia artificial, algoritmos de optimización y arquitecturas distribuidas que sirven de base para este trabajo. Se describe el aprendizaje profundo y las redes neuronales aplicadas a visión computacional. También se abordan conceptos sobre ingeniería de software.
% 
% El \autoref{chapter:sistema_actual} describe el estado de la plataforma legacy, detallando las funcionalidades, la arquitectura y las limitaciones presentes en la implementación original.
% 
% Los Capítulos~\ref{chapter:rediseño} y~\ref{chapter:implementacion} documentan el desarrollo de la nueva arquitectura. Abordan el diseño del sistema distribuido y la implementación de los microservicios, respectivamente.
% 
% En el \autoref{chapter:evaluacion} se presentan los resultados obtenidos en términos de rendimiento y escalabilidad.
% 
% Finalmente, las conclusiones generales son presentadas en el \autoref{chapter:conclusiones}, donde también se discuten posibles líneas futuras de desarrollo.


