\chapter{Introducción}
\label{chapter:intro}

\completar{INTRO}

\section{Objetivos}

\subsection{Objetivo General}
Analizar y comparar diferentes metodologías de valoración para opciones del tipo barrera, abarcando desde soluciones analíticas en tiempo continuo hasta algoritmos de aproximación en modelos discretos.

\subsection{Objetivos Específicos}
\begin{itemize}
    \item Estudiar el modelo de Black-Scholes para la obtención de fórmulas cerradas de opciones barrera europeas.
    \item Implementar y analizar la eficiencia de modelos de árboles (binomial y trinomial) en la valoración de opciones barrera americanas, evaluando el uso de mallas adaptativas.
    \item Explorar la técnica de replicación estática como estrategia de cobertura mediante portafolios de opciones vanilla.
    \item Evaluar las ventajas y limitaciones de cada método en términos de precisión y complejidad computacional.
\end{itemize}

\section{Resultados originales presentados}

\completar{SI HAY RESULTADOS COMO CONFERENCIAS O PAPERS}

\section{Organización del trabajo}

Los capítulos de este trabajo se organizan de la siguiente manera:

En el Capítulo~\ref{chapter:antecedentes} se presentan los fundamentos teóricos de las opciones barrera, el modelo de Black-Scholes para activos con barreras y los conceptos básicos de procesos estocásticos necesarios para su comprensión.

El Capítulo~\ref{chapter:implementacion} describe el estado del arte y las metodologías de valoración propuestas, detallando los modelos discretos de árboles y el uso de mallas adaptativas.

Los Capítulos~\ref{chapter:metodologia} y~\ref{chapter:resultados} documentan el desarrollo de las simulaciones y la replicación estática. Abordan la implementación algorítmica y el análisis comparativo de resultados, respectivamente.

Finalmente, las conclusiones generales son presentadas en el Capítulo~\ref{chapter:conclu}, donde también se discuten posibles líneas futuras de desarrollo.

% =============================================================================
% EJEMPLO COMPLETADO:
% =============================================================================
% Los capítulos de este trabajo se organizan de la siguiente manera:
% 
% En el \autoref{chapter:fundamentos} se presentan los fundamentos de inteligencia artificial, algoritmos de optimización y arquitecturas distribuidas que sirven de base para este trabajo. Se describe el aprendizaje profundo y las redes neuronales aplicadas a visión computacional. También se abordan conceptos sobre ingeniería de software.
% 
% El \autoref{chapter:sistema_actual} describe el estado de la plataforma legacy, detallando las funcionalidades, la arquitectura y las limitaciones presentes en la implementación original.
% 
% Los Capítulos~\ref{chapter:rediseño} y~\ref{chapter:implementacion} documentan el desarrollo de la nueva arquitectura. Abordan el diseño del sistema distribuido y la implementación de los microservicios, respectivamente.
% 
% En el \autoref{chapter:evaluacion} se presentan los resultados obtenidos en términos de rendimiento y escalabilidad.
% 
% Finalmente, las conclusiones generales son presentadas en el \autoref{chapter:conclusiones}, donde también se discuten posibles líneas futuras de desarrollo.


