Esta tesis presenta métodos para la valorización y discretización de opciones barrera, derivados financieros cuyo pago depende de si el precio del activo subyacente cruza un nivel de barrera predeterminado durante la vida de la opción. Las opciones barrera son ampliamente utilizadas en los mercados financieros para la gestión de riesgos y estrategias de inversión debido a su menor costo en comparación con las opciones estándar.

Los objetivos principales de este trabajo son analizar e implementar métodos para la valoración de opciones barrera, con énfasis particular en las técnicas de discretización que capturan con precisión la condición de barrera. La metodología incluye: (1) el modelo continuo de Black-Scholes, que asume un comportamiento lognormal del precio del activo, permitiendo fórmulas cerradas para opciones barrera europeas y sentando las bases para simulaciones de Monte Carlo en el caso americano; (2) modelos discretos como el binomial y trinomial que analizan el conjunto completo de trayectorias posibles, abordando el problema de la subestimación o sobreestimación cuando la barrera no coincide con los nodos del árbol mediante estrategias con mallas adaptativas; y (3) la replicación estática, una técnica para cubrir opciones barrera mediante la construcción de un portafolio de opciones vanillas que reproduzca el mismo payoff bajo ciertas condiciones. Nos enfocamos tanto en opciones barrera de estilo europeo como americano.

Los resultados demuestran la efectividad de los esquemas de discretización propuestos para lograr valoraciones precisas manteniendo la eficiencia computacional. Comparamos el rendimiento de diferentes enfoques numéricos y analizamos sus propiedades de convergencia. El trabajo contribuye a la comprensión de métodos numéricos para la valoración de opciones exóticas y proporciona herramientas prácticas para la evaluación de riesgos financieros.

\textbf{Palabras clave:} opciones barrera, valoración de opciones, modelo de Black-Scholes, árboles binomiales y trinomiales, replicación estática, mallas adaptativas, derivados financieros.
