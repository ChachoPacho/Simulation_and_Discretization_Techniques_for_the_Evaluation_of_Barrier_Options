\chapter{Resultados y Comparación}
\label{chapter:resultados}

% OBJETIVO: Comparar los 4 métodos
% Este es el CORAZÓN de tu tesis

\section{Diseño de Experimentos}
\label{sec:diseno_experimentos}

% QUÉ ESCRIBIR AQUÍ:
% - Casos de prueba definidos
% - Métricas de evaluación

\subsection{Casos de Prueba para Opciones Europeas}

% DEFINIR BATERÍA DE CASOS:
% Variar: S, K, H, σ, r, T
% 
% Tabla ejemplo:
% Caso | Tipo | S | K | H | σ | r | T | Descripción
% 1    | DOC  | 100 | 100 | 90 | 0.20 | 0.05 | 1.0 | S lejos de H
% 2    | DOC  | 92  | 100 | 90 | 0.20 | 0.05 | 1.0 | S cerca de H
% ...

\subsection{Casos de Prueba para Opciones Americanas}

% DEFINIR CASOS para Monte Carlo

\subsection{Métricas de Evaluación}

% Definir:
% - Error absoluto: |V_método - V_R&R|
% - Error relativo: |V_método - V_R&R| / V_R&R
% - Tiempo de cómputo
% - Eficiencia: precisión / tiempo

\section{Resultados para Opciones Europeas}
\label{sec:resultados_europeas}

% QUÉ ESCRIBIR AQUÍ:
% Comparar: R&R vs Árboles+Mallas vs Replicante

\subsection{Tablas Comparativas}

% INCLUIR TABLA GRANDE:
% Caso | R&R | Árboles | Error | Tiempo | Replicante | Error | Tiempo

\subsection{Análisis de Precisión}

% Analizar:
% - Qué método es más preciso
% - En qué casos cada método funciona mejor
% - Casos problemáticos

\subsection{Análisis de Tiempo Computacional}

% Analizar:
% - Qué método es más rápido
% - Trade-off precisión vs tiempo
% - Gráficos scatter: tiempo vs error

\subsection{Comportamiento Cerca de la Barrera}

% Analizar:
% - Qué pasa cuando S está muy cerca de H
% - Qué método maneja mejor esta situación
% - Problemas numéricos observados

\subsection{Gráficos Comparativos}

% INCLUIR GRÁFICOS:
% - Error vs parámetros (S, σ, T)
% - Tiempo vs parámetros
% - Convergencia de cada método

\section{Resultados para Opciones Americanas}
\label{sec:resultados_americanas}

% QUÉ ESCRIBIR AQUÍ:
% Resultados de Monte Carlo

\subsection{Tablas de Resultados}

% INCLUIR TABLA:
% Caso | Valor MC | IC 95% | Tiempo | N_simulaciones

\subsection{Análisis de Convergencia}

% Mostrar:
% - Convergencia con número de simulaciones
% - Convergencia con número de pasos temporales

\subsection{Comparación con Europeas}

% Comparar:
% - Valor americano vs europeo (límite superior)
% - Prima de ejercicio anticipado

\section{Comparación Global de Métodos}
\label{sec:comparacion_global}

% QUÉ ESCRIBIR AQUÍ:
% Síntesis de todos los resultados

\subsection{Tabla Resumen}

% CREAR TABLA RESUMEN:
% Método | Precisión | Velocidad | Flexibilidad | Casos de Uso

\subsection{Análisis de Casos de Uso Óptimos}

% Identificar:
% - R&R: mejor cuando... (referencia, muy rápido)
% - Árboles+Mallas: mejor cuando... (buena precisión, flexible)
% - Replicante: mejor cuando... (cobertura, no requiere rebalanceo)
% - Monte Carlo: mejor cuando... (americanas, único método)

\subsection{Recomendaciones Prácticas}

% DAR GUÍAS CONCRETAS:
% - Si necesitas velocidad → usar R&R
% - Si S está cerca de H → usar Árboles+Mallas
% - Si necesitas cobertura → usar Replicante
% - Si es americana → usar Monte Carlo
% - Para producción → usar X
% - Para análisis de riesgo → usar Y

\section{Discusión}
\label{sec:discusion}

% QUÉ ESCRIBIR AQUÍ:
% Interpretación de resultados

\subsection{Ventajas y Desventajas Observadas}

% Para cada método:
% - Ventajas principales
% - Desventajas principales
% - Sorpresas o hallazgos inesperados

\subsection{Limitaciones del Estudio}

% Mencionar:
% - Qué no se pudo hacer
% - Supuestos mantenidos
% - Casos no considerados
