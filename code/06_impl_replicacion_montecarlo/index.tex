\chapter{Implementación - Replicación Estática y Monte Carlo}
\label{chapter:impl_replicacion_mc}

% OBJETIVO: Documentar implementación de MÉTODO 3 y MÉTODO 4

\section{Replicación Estática}
\label{sec:impl_replicacion}

% QUÉ ESCRIBIR AQUÍ:
% Implementación del portafolio replicante (MÉTODO 3)

\subsection{Descripción del Algoritmo}

% Explicar:
% - Cómo construyes el portafolio
% - Selección de strikes
% - Cálculo de pesos

\subsubsection{Selección de Strikes}

% Explicar:
% - Qué strikes eliges para las opciones vanilla
% - Criterio de selección
% - Número de opciones en el portafolio

\subsubsection{Cálculo de Pesos}

% Explicar:
% - Cómo calculas el peso de cada opción
% - Fórmulas utilizadas
% - Normalización

\subsection{Detalles de Implementación}

% Explicar:
% - Lenguaje y librerías
% - Código relevante

\subsection{Validación del Portafolio Replicante}

% QUÉ ESCRIBIR AQUÍ:
% - Comparar con R&R
% - Análisis de error de replicación
% - Ejemplos numéricos

\subsubsection{Comparación con Rubinstein-Reiner}

% INCLUIR TABLA:
% Caso | R&R | Replicante | Error | N_opciones

\subsubsection{Análisis de Convergencia}

% Mostrar:
% - Error vs número de opciones en portafolio
% - Gráfico de convergencia

\subsubsection{Ejemplos Numéricos}

% INCLUIR 2-3 CASOS:
% - Down-and-Out Call
% - Up-and-In Put
% - Mostrar composición del portafolio

\section{Monte Carlo para Opciones Americanas}
\label{sec:impl_montecarlo}

% QUÉ ESCRIBIR AQUÍ:
% Implementación de Monte Carlo (MÉTODO 4)

\subsection{Descripción del Algoritmo}

% Explicar:
% - Generación de trayectorias
% - Monitoreo de barrera
% - Método de Longstaff-Schwartz

\subsubsection{Generación de Trayectorias}

% Explicar:
% - Fórmula: S_{t+Δt} = S_t exp[(r-σ²/2)Δt + σ√Δt·Z]
% - Número de pasos temporales
% - Número de simulaciones M

\subsubsection{Monitoreo de Barrera}

% Explicar:
% - Frecuencia de monitoreo
% - Qué pasa cuando se toca la barrera
% - Knock-in vs Knock-out

\subsubsection{Método de Longstaff-Schwartz}

% Explicar:
% - Regresión para estimar valor de continuación
% - Decisión de ejercicio en cada nodo
% - Algoritmo backward

\subsection{Detalles de Implementación}

% Explicar:
% - Lenguaje y librerías
% - Generador de números aleatorios
% - Código relevante

\subsubsection{Código Relevante}

% INCLUIR CÓDIGO:
% - Generación de trayectorias
% - Algoritmo LSM
% - Decisión de ejercicio

\subsection{Validación de Monte Carlo}

% QUÉ ESCRIBIR AQUÍ:
% - Comparar con literatura (si hay benchmarks)
% - Análisis de convergencia
% - Intervalos de confianza

\subsubsection{Análisis de Convergencia}

% Mostrar:
% - Error vs número de simulaciones
% - Error vs número de pasos temporales
% - Gráficos

\subsubsection{Intervalos de Confianza}

% Explicar:
% - Cálculo de IC al 95%
% - Error estándar: SE = σ̂/√M

\subsubsection{Ejemplos Numéricos}

% INCLUIR 2-3 CASOS:
% - American Down-and-Out Call
% - American Up-and-In Put
% - Comparar con europeas (límite superior)

\subsection{Técnicas de Reducción de Varianza}

% SI LAS USAS:
% - Variables antitéticas
% - Variables de control
% - Impacto en precisión y tiempo
