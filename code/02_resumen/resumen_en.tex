This thesis presents methods for the valuation and discretization of barrier options, financial derivatives whose payoff depends on whether the underlying asset price crosses a predetermined barrier level during the option's lifetime. Barrier options are widely used in financial markets for risk management and investment strategies due to their lower cost compared to standard options.

The main objectives of this work are to analyze and implement methods for pricing barrier options, with particular emphasis on discretization techniques that accurately capture the barrier condition. The methodology includes: (1) the continuous Black-Scholes model, which assumes a lognormal behavior of the asset price, allowing closed-form formulas for European barrier options and providing the foundation for Monte Carlo simulations in the American case; (2) discrete models such as binomial and trinomial trees that analyze the complete set of possible trajectories, addressing the problem of underestimation or overestimation when the barrier does not align with tree nodes through strategies using adaptive meshes; and (3) static replication, a technique for hedging barrier options by constructing a portfolio of vanilla options that reproduces the same payoff under certain conditions. We focus on both European and American-style barrier options.

The results demonstrate the effectiveness of the proposed discretization schemes in achieving accurate valuations while maintaining computational efficiency. We compare the performance of different numerical approaches and analyze their convergence properties. The work contributes to the understanding of numerical methods for exotic options pricing and provides practical tools for financial risk assessment.

\textbf{Keywords:} barrier options, option pricing, Black-Scholes model, binomial and trinomial trees, static replication, adaptive meshes, financial derivatives.
