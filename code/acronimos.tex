% =============================================================================
% DOCUMENTACIÓN DEL PAQUETE GLOSSARIES
% =============================================================================

% DEFINICIÓN DE ACRÓNIMOS
% Sintaxis: \newacronym{etiqueta}{forma_corta}{forma_larga}
% \newacronym{F77}{F77}{FORTRAN-77}
% \newacronym{F90}{F90}{Fortran 90}
% \newacronym{OP}{OP}{optimización de portafolio}

% COMANDOS DISPONIBLES
% \gls{F77} -> Muestra la forma "inteligente" del acrónimo. En la primera aparición muestra la forma completa, en usos posteriores solo la forma corta 
% \acrlong{F77} -> Siempre muestra la forma larga/expandida del acrónimo FORTRAN-77
% \acrshort{F77} -> Siempre muestra la forma corta/abreviada del acrónimo F77
% \acrfull{F77} -> Siempre muestra la forma completa (forma larga seguida de la forma corta entre paréntesis) FORTRAN-77 (F77)


