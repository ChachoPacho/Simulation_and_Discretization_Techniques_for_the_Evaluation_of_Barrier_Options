\chapter{Implementación - Árboles Trinomiales con Mallas Adaptativas}
\label{chapter:impl_arboles_mallas}

% OBJETIVO: Documentar tu implementación del método híbrido
% Este es tu MÉTODO 2

\section{Descripción del Algoritmo}
\label{sec:algoritmo_arboles_mallas}

% QUÉ ESCRIBIR AQUÍ:
% - Pseudocódigo del algoritmo completo
% - Cómo integras árboles con mallas adaptativas

\subsection{Construcción del Árbol Trinomial}

% Explicar:
% - Parámetros: u, d, p_u, p_m, p_d
% - Construcción de nodos
% - Número de pasos temporales

\subsection{Adaptación de la Malla}

% Explicar:
% - Criterios de refinamiento cerca de barrera
% - Cómo ajustas la distribución de nodos
% - Concentración cerca de H y K

\subsection{Valoración Backward}

% Explicar:
% - Algoritmo de valoración desde T hasta 0
% - Manejo de la barrera en cada nodo
% - Interpolación si es necesario

\section{Detalles de Implementación}
\label{sec:detalles_impl_arboles}

% QUÉ ESCRIBIR AQUÍ:
% - Lenguaje y librerías
% - Estructura de datos
% - Código relevante

\subsection{Estructura de Datos}

% Explicar:
% - Cómo representas el árbol
% - Cómo almacenas valores en nodos
% - Cómo manejas la malla adaptativa

\subsection{Código Relevante}

% INCLUIR CÓDIGO:
% - Construcción del árbol
% - Refinamiento de malla
% - Valoración backward

\section{Validación}
\label{sec:validacion_arboles}

% QUÉ ESCRIBIR AQUÍ:
% - Comparar con fórmulas de R&R
% - Análisis de convergencia
% - Ejemplos numéricos

\subsection{Comparación con Rubinstein-Reiner}

% INCLUIR TABLA:
% Caso | R&R | Árboles+Mallas | Error | Tiempo

\subsection{Análisis de Convergencia}

% Mostrar:
% - Error vs número de pasos temporales
% - Error vs número de nodos espaciales
% - Gráficos de convergencia

\subsection{Ejemplos Numéricos}

% INCLUIR 3-4 CASOS:
% - Down-and-Out Call con S cerca de H
% - Down-and-Out Call con S lejos de H
% - Up-and-In Put
% - Otro caso
