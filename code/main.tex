%\documentclass[spanish,a4paper,11pt,oneside,links]{report}
\documentclass[spanish,a4paper,oneside,10pt,openright]{book}
\usepackage{graphicx}
\usepackage{amsmath,amsfonts,amssymb}
\usepackage[utf8]{inputenc}
\usepackage[spanish, es-tabla]{babel}
\usepackage[Algoritmo]{algorithm}
\usepackage{algpseudocode}
\usepackage{url}
\usepackage{fancyhdr}
\usepackage{epstopdf}
\usepackage{minted}
\usepackage{csquotes}
\usepackage{rotating}
\usepackage{etoolbox}
\usepackage{zref-xr}
\usepackage{amssymb}
\usepackage{hyperref}
\usepackage[authoryear,round,longnamesfirst]{natbib}
\usepackage{multicol}
\usepackage{spverbatim}
\usepackage[pdf]{graphviz}
\usepackage{dirtree}
\usepackage{tabularx}
    \newcolumntype{L}{>{\raggedright\arraybackslash}X}
\usepackage{emptypage}
\usepackage[acronym,shortcuts,xindy]{glossaries}
\usepackage{markdown}
\usepackage{booktabs}
\usepackage{xcolor}
\usepackage{changepage}
\usepackage{tcolorbox}
\usepackage{listings}
\usepackage{datetime}

\newcommand{\capitalmonthname}[1]{%
  \ifcase#1\or
    Enero\or Febrero\or Marzo\or Abril\or Mayo\or Junio\or
    Julio\or Agosto\or Septiembre\or Octubre\or Noviembre\or Diciembre
  \fi
}

\newdateformat{monthyeardate}{%
  \capitalmonthname{\THEMONTH}, \THEYEAR}

\newcommand{\tfcTitle}{Técnicas de simulación y discretización para la valoración de opciones barrera}
\newcommand{\tfcAuthor}{Gonzalo A. Bordón}
\newcommand{\tfcAdvisors}{Dra. Noemí Patricia Kisbye}
\newcommand{\tfcGrade}{Licenciatura en Ciencias de la Computación}
\newcommand{\tfcFaculty}{Facultad De Matemática, Astronomía, Física Y Computación}
\newcommand{\tfcCollege}{Universidad Nacional de Córdoba}

\setlength{\headheight}{13.07225pt}

% wide page for side by side figures, tables, etc
\newlength{\offsetpage}
\setlength{\offsetpage}{1.0cm}
\newenvironment{widepage}{\begin{adjustwidth}{-\offsetpage}{-\offsetpage}%
    \addtolength{\textwidth}{2\offsetpage}}%
{\end{adjustwidth}}

\newtheorem{definition}{Definición}

\renewcommand{\listalgorithmname}{Índice de Algoritmos}

\addto\extrasspanish{
    \renewcommand{\sectionautorefname}{Sec.}
}


\DeclareMathOperator*{\argmax}{arg\,max}  % in your preamble
\DeclareMathOperator*{\argmin}{arg\,min}  % in your preamble

\DeclareMathOperator*{\minimize}{minimize}
\DeclareMathOperator*{\maximize}{maximize}

\makeglossaries

% =============================================================================
% DOCUMENTACIÓN DEL PAQUETE GLOSSARIES
% =============================================================================

% DEFINICIÓN DE ACRÓNIMOS
% Sintaxis: \newacronym{etiqueta}{forma_corta}{forma_larga}
% \newacronym{F77}{F77}{FORTRAN-77}
% \newacronym{F90}{F90}{Fortran 90}
% \newacronym{OP}{OP}{optimización de portafolio}

% COMANDOS DISPONIBLES
% \gls{F77} -> Muestra la forma "inteligente" del acrónimo. En la primera aparición muestra la forma completa, en usos posteriores solo la forma corta 
% \acrlong{F77} -> Siempre muestra la forma larga/expandida del acrónimo FORTRAN-77
% \acrshort{F77} -> Siempre muestra la forma corta/abreviada del acrónimo F77
% \acrfull{F77} -> Siempre muestra la forma completa (forma larga seguida de la forma corta entre paréntesis) FORTRAN-77 (F77)




\lhead[]{\rightmark}
\chead[]{}
\rhead[\leftmark]{}
\renewcommand{\headrulewidth}{0.5pt}

\fancypagestyle{plain}{
\fancyhead[L]{}
\fancyhead[C]{}
\fancyhead[R]{}
\renewcommand{\headrulewidth}{0pt}
}

\pagestyle{fancy}

\widowpenalty 10000
\clubpenalty 10000

% AGREGAGO POR JUAN =====================================================
% COMANDOS DE CORRECCION
\newcommand{\jbc}[1]{{\color{olive} #1 (JBC)\ }}

\newcommand{\rev}[1]{{\color{blue} #1\ }}

\newcommand{\citeme}[1]{{\color{red}{¡¡CITEME: #1!!}}}
\newcommand{\completar}[1]{{\color{red!75}{Completar: #1}}}

\renewcommand{\listingscaption}{Algoritmo}
\renewcommand{\listoflistingscaption}{Lista de algoritmos}

%\usemintedstyle{algol_nu}
%https://pygments.org/styles/

\setminted{
    breaklines=true
}

\renewcommand{\listingscaption}{Código}
\renewcommand{\thelisting}{\thesection.\arabic{listing}}
\preto\section{\setcounter{listing}{0}}

\begin{document}



% % Caratula
\begin{titlepage}

% LOGOS SUPERIORES
\noindent
\begin{minipage}[t]{0.5\linewidth}
    \flushleft
    \includegraphics[width=6cm]{00_tapa/figures/logo_unc.png} % Descomentar si tienes la imagen
\end{minipage}%
\begin{minipage}[t]{0.5\linewidth}
    \flushright
    \includegraphics[width=6cm]{00_tapa/figures/logo_famaf.png} % Descomentar si tienes la imagen
\end{minipage}

\vspace{3cm} % Espacio vertical para separar

% TÍTULO Y AUTOR
\begin{center}
    {\LARGE \textbf{\tfcTitle}}\\[1.5cm]
    {\large por}\\[0.5cm]
    {\Large \textbf{\tfcAuthor}}
\end{center}

\vspace{1.5cm} 

% TEXTO DESCRIPTIVO
\noindent
Presentado ante la \textbf{\MakeUppercase{\tfcFaculty}} como parte de los requerimientos para la obtención del grado de \tfcGrade de la

\begin{center}
    \textbf{\MakeUppercase{\tfcCollege}}
\end{center}

\vspace{1cm}

\noindent
\begin{center}
    \monthyeardate\today
\end{center}

\vspace{0.5cm}

% DIRECTOR/A
\noindent
\begin{center}
Directores: \tfcAdvisors
\end{center}

\vfill % Espacio hasta el final de la página

% LICENCIA CREATIVE COMMONS
\begin{center}
    \includegraphics[width=3cm]{00_tapa/figures/ccby40.png}\\    \vspace{0.3cm}
    Generación y diseño de herramientas para el análisis de retornos de carteras de inversión artificiales y reales © 2025 by Diego Nicolas Gimenez Irusta is licensed under
    \href{https://creativecommons.org/licenses/by-sa/4.0/?ref=chooser-v1}{Creative Commons Attribution-ShareAlike 4.0 International}
\end{center}



\end{titlepage}
 \cleardoublepage


\pagenumbering{gobble}
Certifico que el trabajo incluido en este documento es el resultado de tareas de investigación
originales y que no ha sido presentado para optar a un título en ninguna otra
Universidad o Institución.

\vspace{3cm}
\begin{flushright}
\tfcAuthor
\end{flushright}
\cleardoublepage

% RECONOCIMIENTS + AGRADECIMIENTOS ============================================

%\chapter*{Reconocimientos}
%\pagenumbering{Roman} % para comenzar la numeracion de paginas en numeros romanos
%\addcontentsline{toc}{chapter}{Reconocimientos} % si queremos que aparezca en el índice
%\markboth{RECONOCIMIENTOS}{RECONOCIMIENTOS} % encabezado
%Esta Tesis fue desarrollada gracias a la ayuda económica de ...


%\phantom{p. 1}
\clearpage
\thispagestyle{empty}
\phantom{p. 2}
\clearpage

\chapter*{Agradecimientos} % si no queremos que añada la palabra "Capitulo"
\pagenumbering{Roman} % para comenzar la numeracion de paginas en numeros romanos
\addcontentsline{toc}{chapter}{Agradecimientos} % si queremos que aparezca en el índice
\markboth{AGRADECIMIENTOS}{AGRADECIMIENTOS} % encabezado
\input{./01_agradecimientos/agradecimientos.tex}

% ABSTRACT ====================================================================

\phantom{p. 1}
\clearpage
\thispagestyle{empty}
\phantom{p. 2}
\clearpage

\chapter*{Resumen} % si no queremos que añada la palabra "Capitulo"
\addcontentsline{toc}{chapter}{Resumen} % si queremos que aparezca en el índice
\markboth{RESUMEN}{RESUMEN} % encabezado
Esta tesis presenta métodos para la valorización y discretización de opciones barrera, derivados financieros cuyo pago depende de si el precio del activo subyacente cruza un nivel de barrera predeterminado durante la vida de la opción. Las opciones barrera son ampliamente utilizadas en los mercados financieros para la gestión de riesgos y estrategias de inversión debido a su menor costo en comparación con las opciones estándar.

Los objetivos principales de este trabajo son analizar e implementar métodos para la valoración de opciones barrera, con énfasis particular en las técnicas de discretización que capturan con precisión la condición de barrera. La metodología incluye: (1) el modelo continuo de Black-Scholes, que asume un comportamiento lognormal del precio del activo, permitiendo fórmulas cerradas para opciones barrera europeas y sentando las bases para simulaciones de Monte Carlo en el caso americano; (2) modelos discretos como el binomial y trinomial que analizan el conjunto completo de trayectorias posibles, abordando el problema de la subestimación o sobreestimación cuando la barrera no coincide con los nodos del árbol mediante estrategias con mallas adaptativas; y (3) la replicación estática, una técnica para cubrir opciones barrera mediante la construcción de un portafolio de opciones vanillas que reproduzca el mismo payoff bajo ciertas condiciones. Nos enfocamos tanto en opciones barrera de estilo europeo como americano.

Los resultados demuestran la efectividad de los esquemas de discretización propuestos para lograr valoraciones precisas manteniendo la eficiencia computacional. Comparamos el rendimiento de diferentes enfoques numéricos y analizamos sus propiedades de convergencia. El trabajo contribuye a la comprensión de métodos numéricos para la valoración de opciones exóticas y proporciona herramientas prácticas para la evaluación de riesgos financieros.

\textbf{Palabras clave:} opciones barrera, valoración de opciones, modelo de Black-Scholes, árboles binomiales y trinomiales, replicación estática, mallas adaptativas, derivados financieros.


\phantom{p. 1}
\clearpage
\thispagestyle{empty}

\chapter*{Abstract} % si no queremos que añada la palabra "Capitulo"
\addcontentsline{toc}{chapter}{Abstract} % si queremos que aparezca en el índice
\markboth{ABSTRACT}{ABSTRACT} % encabezado
This thesis presents methods for the valuation and discretization of barrier options, financial derivatives whose payoff depends on whether the underlying asset price crosses a predetermined barrier level during the option's lifetime. Barrier options are widely used in financial markets for risk management and investment strategies due to their lower cost compared to standard options.

The main objectives of this work are to analyze and implement methods for pricing barrier options, with particular emphasis on discretization techniques that accurately capture the barrier condition. The methodology includes: (1) the continuous Black-Scholes model, which assumes a lognormal behavior of the asset price, allowing closed-form formulas for European barrier options and providing the foundation for Monte Carlo simulations in the American case; (2) discrete models such as binomial and trinomial trees that analyze the complete set of possible trajectories, addressing the problem of underestimation or overestimation when the barrier does not align with tree nodes through strategies using adaptive meshes; and (3) static replication, a technique for hedging barrier options by constructing a portfolio of vanilla options that reproduces the same payoff under certain conditions. We focus on both European and American-style barrier options.

The results demonstrate the effectiveness of the proposed discretization schemes in achieving accurate valuations while maintaining computational efficiency. We compare the performance of different numerical approaches and analyze their convergence properties. The work contributes to the understanding of numerical methods for exotic options pricing and provides practical tools for financial risk assessment.

\textbf{Keywords:} barrier options, option pricing, Black-Scholes model, binomial and trinomial trees, static replication, adaptive meshes, financial derivatives.


\clearpage
\phantom{p. 1}
\thispagestyle{empty}



% GLOSARIO=====================================================================

\renewcommand*\glspostdescription{\dotfill}
\setlength{\glslistdottedwidth}{1\textwidth}
\setlength{\glsdescwidth}{0.89\textwidth}
\markboth{Glosario}{Glosario} % encabezado
%\glsaddall
\printglossary[type=\acronymtype,title=Glosario]
\addcontentsline{toc}{chapter}{Glosario}


% CONTENIDOS ==================================================================
\phantom{p. 1}
\clearpage

\tableofcontents % indice de contenidos

\phantom{p. 1}
\clearpage
\thispagestyle{empty}
\phantom{p. 2}
\clearpage

% FIGURES =====================================================================

\cleardoublepage
\addcontentsline{toc}{chapter}{Índice de figuras} % para que aparezca en el indice de contenidos
\listoffigures % indice de figuras

\phantom{p. 1}
\clearpage
\thispagestyle{empty}
\phantom{p. 1}
\clearpage

% TABLES ======================================================================

% \cleardoublepage
% \addcontentsline{toc}{chapter}{Índice de tablas} % para que aparezca en el indice de contenidos
% \listoftables % indice de tablas

% pagina en blanco
\newpage

\cleardoublepage
\renewcommand\listoflistingscaption{Índice de códigos}
\listoflistings % Now typeset the list
\addcontentsline{toc}{chapter}{Índice de código} % para que aparezca en el indice de contenidos

% pagina en blanco
\newpage
\pagenumbering{arabic}
\phantom{p. 1}
\clearpage


% CONTENT =====================================================================


\chapter{Introducción}
\label{chapter:intro}

\completar{INTRO}

\section{Resultados originales presentados}

\completar{SI HAY RESULTADOS COMO CONFERENCIAS O PAPERS}

\section{Organización del trabajo}

\section{Organización del trabajo}

Los capítulos de este trabajo se organizan de la siguiente manera:

En el Capítulo~\ref{chapter:antecedentes} se 
presentan los fundamentos de sistemas de riego en general, su uso, y por que importante detectarlos, y como se esta trabajando actualmente con redes neuronales para lograrlo.

El Capítulo~\ref{chapter:} describe el estado de [SISTEMA\_OBJETO], detallando las funcionalidades, la arquitectura y las limitaciones presentes en [IMPLEMENTACION\_ACTUAL].

Los Capítulos~\ref{chapter:} y~\ref{chapter:} documentan el desarrollo de [NUEVA\_VERSION]. Abordan [ASPECTO\_TECNICO\_1] y [ASPECTO\_TECNICO\_2], respectivamente.

En el Capítulo~\ref{chapter:} se presentan los resultados obtenidos en términos de [METRICA\_1] y [METRICA\_2].

Finalmente, las conclusiones generales son presentadas en el Capítulo~\ref{chapter:conclu}, donde también se discuten posibles líneas futuras de desarrollo.

% =============================================================================
% EJEMPLO COMPLETADO:
% =============================================================================
% Los capítulos de este trabajo se organizan de la siguiente manera:
% 
% En el \autoref{chapter:fundamentos} se presentan los fundamentos de inteligencia artificial, algoritmos de optimización y arquitecturas distribuidas que sirven de base para este trabajo. Se describe el aprendizaje profundo y las redes neuronales aplicadas a visión computacional. También se abordan conceptos sobre ingeniería de software.
% 
% El \autoref{chapter:sistema_actual} describe el estado de la plataforma legacy, detallando las funcionalidades, la arquitectura y las limitaciones presentes en la implementación original.
% 
% Los Capítulos~\ref{chapter:rediseño} y~\ref{chapter:implementacion} documentan el desarrollo de la nueva arquitectura. Abordan el diseño del sistema distribuido y la implementación de los microservicios, respectivamente.
% 
% En el \autoref{chapter:evaluacion} se presentan los resultados obtenidos en términos de rendimiento y escalabilidad.
% 
% Finalmente, las conclusiones generales son presentadas en el \autoref{chapter:conclusiones}, donde también se discuten posibles líneas futuras de desarrollo.


 \cleardoublepage % 3 pag
\chapter{Antecedentes}
\label{chapter:antecedentes}

\section{Opciones Barrera}
Las opciones tipo barrera son derivados financieros cuya existencia depende de si el precio del activo subyacente cruza o no un nivel de precio predeterminado (la barrera) durante el periodo de vida de la opción.

\subsection{Clasificación y Payoffs}
\begin{itemize}
    \item \textbf{Knock-in}: La opción vale sólo si se alcanzó la barrera.
    \begin{itemize}
        \item Up-and-In (UI): La barrera $B$ está por encima del precio inicial $S_0$.
        \item Down-and-In (DI): La barrera $B$ está por debajo del precio inicial $S_0$.
    \end{itemize}
    \item \textbf{Knock-out}: La opción deja de valer si se alcanzó la barrera.
    \begin{itemize}
        \item Up-and-Out (UO): La barrera $B$ está por encima del precio inicial $S_0$.
        \item Down-and-Out (DO): La barrera $B$ está por debajo del precio inicial $S_0$.
    \end{itemize}
\end{itemize}

Formalmente, el payoff de una opción barrera europea con vencimiento $T$, precio de ejercicio $K$ y barrera $B$ puede expresarse utilizando la función indicadora $\mathbb{I}_{A}$. Por ejemplo, para una opción \textit{Down-and-Out Call}:

\begin{equation}
    \text{Payoff}_{DO} = \max(S_T - K, 0) \cdot \mathbb{I}_{\{\min_{0 \leq t \leq T} S_t > B\}}
\end{equation}

Y para una opción \textit{Down-and-In Call}:

\begin{equation}
    \text{Payoff}_{DI} = \max(S_T - K, 0) \cdot \mathbb{I}_{\{\min_{0 \leq t \leq T} S_t \leq B\}}
\end{equation}

De manera análoga se definen los payoffs para las variantes \textit{Up} considerando el máximo del proceso en lugar del mínimo: $\mathbb{I}_{\{\max_{0 \leq t \leq T} S_t < B\}}$ para \textit{Up-and-Out} y $\mathbb{I}_{\{\max_{0 \leq t \leq T} S_t \geq B\}}$ para \textit{Up-and-In}.

\section{Dinámica de Precios y Modelado}
\subsection{Movimiento Browniano Geométrico}
El modelo estándar de valoración asume que el precio del activo subyacente $S_t$ sigue un proceso...

\subsection{Primer Tiempo de Pasaje}
Para las opciones barrera, es fundamental el estudio del primer instante en que el proceso alcanza el nivel $B$...

\completar{Desarrollar definiciones matemáticas y fórmulas de payoff}
 \cleardoublepage % 30 - 40 pag
% \input{./05_implementacion/05_implementacion.tex} \cleardoublepage % 10
% \input{./06_resultados/06_resultados.tex} \cleardoublepage % ~5
\chapter{Conclusiones y trabajo a futuro}
\label{chapter:conclu}

\completar{CONCLUSIONES}



 \cleardoublepage


% APENDICES====================================================================


% \appendix



% BIBLIOGRAFIA ================================================================

\phantom{p. 1}
\clearpage
\cleardoublepage
\addcontentsline{toc}{chapter}{Bibliografía}
\bibliographystyle{apalike}
\bibliography{tesis}


\end{document}
